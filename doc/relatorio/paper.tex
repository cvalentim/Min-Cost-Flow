\documentclass[12pt]{article}

\usepackage[brazilian]{babel}
\usepackage[utf8]{inputenc}

% This first part of the file is called the PREAMBLE. It includes
% customizations and command definitions. The preamble is everything
% between \documentclass and \begin{document}.

\usepackage[margin=1in]{geometry}  % set the margins to 1in on all sides
\usepackage{graphicx}              % to include figures
\usepackage{amsmath}               % great math stuff
\usepackage{amsfonts}              % for blackboard bold, etc
\usepackage{amsthm}                % better theorem environments


% various theorems, numbered by section

\newtheorem{thm}{Theorem}[section]
\newtheorem{lem}[thm]{Lemma}
\newtheorem{prop}[thm]{Proposition}
\newtheorem{cor}[thm]{Corollary}
\newtheorem{conj}[thm]{Conjecture}

\DeclareMathOperator{\id}{id}

\newcommand{\bd}[1]{\mathbf{#1}}  % for bolding symbols
\newcommand{\RR}{\mathbb{R}}      % for Real numbers
\newcommand{\ZZ}{\mathbb{Z}}      % for Integers
\newcommand{\col}[1]{\left[\begin{matrix} #1 \end{matrix} \right]}
\newcommand{\comb}[2]{\binom{#1^2 + #2^2}{#1+#2}}


\begin{document}


\nocite{*}

\title{Implementações do Fluxo a Custo Mínimo}

\author{Caio Dias Valentim \\ 
Otimização Combinatória \\
Departamento de Informática - PUC-Rio \\
}

\maketitle

\begin{abstract}
	Neste trabalho investigamos implementações para algoritmos 
	clássicos de fluxo a custo mínimo. A partir das implementações, 
	comparamos a eficiência prática desses algoritmos.
\end{abstract}


\section{Introdução}

O problema de fluxo a custo mínimo é problema bastante estudado na literatura de algoritmos.
Além de ser um problema conceitualmente interessante, ele modela diversas situações práticas.
Dessa forma, diversos algoritmos foram desenvolvidos ao longo dos anos para o problema.

Nesse trabalho, estudamos os algoritmos: \emph{Cycle Canceling} e \emph{Successive Shortest Path}. 
Para o \emph{Cycle Canceling} estudamos duas variações, uma em que o ciclo escolhido é arbitrário e
a outra onde o ciclo de média é sempre escolhido. Da mesma forma, para o \emph{Sucessive Shortest Path} 
estudamos a versão em que os caminhos são arbitrários e a que caminhos são escolhidos a partir de um 
limite $\Delta$.

Nas seções abaixo, definimos o problema formalmente, detalhamos os algoritmos estudados, mostramos
os detalhes da nossa implementação e das instâncias usadas para teste e, por fim, mostramos os experimentos
realizados e tiramos conclusões dos mesmos.

\section{Formalização do Problema}
O problema do fluxo a custo mínimo consiste, basicamente, em encontrar a forma mais barata
distribuir fluxo em uma rede de forma a satisfazer um conjunto de demandas e ofertas.
Um fator de dificuldade é que a rede em questão tem restrições de com capacidades e custos nas arestas.

Para formalizar o problema precisamos de um terminologia:

Dado um grafo direcionado $G(V, E)$, definido por um conjunto de vértices(nós) $V$ e um conjunto
de arestas(arcos) $E$. Para todo arco $(i, j) \in E$ existe um capacidade associada $\mu_{ij}$ que 
determina a maior quantidade de fluxo que pode passar por essa aresta. Além disso, cada aresta
$(i, j) \in E$ apresenta um custo $c_{ij}$ que representa o custo de passar uma unidade de fluxo por essa aresta.

Cada vértice $i \in V$ tem associado um número $b_i$. Esse número representa a demanda oferta do vértice. Se
$b_i > 0$, o nó $i$ é um nó de oferta, se $b_i < 0$ o nó $i$ é de demanda.

Com a terminologia em mente, o problema do fluxo a custo mínimo pode ser representado formalmente pelo problema 
de otimização abaixo:

$$
\min z(x) = \sum_{(i,j)\in E} c_{ij} x_{ij}
$$
$
\text{     sujeito:} 
$
$$
\sum_{j:(i,j)\in E} x_{ij} - \sum_{j:(j,i) \in E} x_{ji} = b_i \text{ ,para todo } i \in V 
$$
$$
0 \le x_{ij} \le \mu_{ij} \text{, para todo } (i, j) \in E
$$	


\section{Algoritmos}
Nessa seção apresentamos uma visão dos algoritmos implementados.

\subsection{Cycle Canceling}
Antes de explicar o algoritmo, vamos enunciar o lema principal que garante sua corretude. 

\begin{lem}[Critério de Otimalizadade de Ciclos Negativos]
	Seja x* uma solução viável para o problema do fluxo a custo mínimo. Então, 
	x* é ótima se, e somente se, a rede residual $G_{x*}$ não contém ciclos negativos.
\end{lem}

Bom, em vista do lema acima, o algoritmo consiste em começar uma solução viável x 
para o problema e, a cada passo, tentar achar um ciclo negativo na rede residual $G_x$.
Se um ciclo negativo for encontrado, passamos o máximo de fluxo possível por esse ciclo 
e atualizamos x de acordo. Se não existir um ciclo, então x é uma soluções ótima.

Note que na descrição do algoritmos dois pontos não totalmente definidos: Como achar 
uma solução viável inicial e como achar um ciclo negativo. Bom, para primeira parte 
questão, podemos achar uma solução viável com um algoritmo de fluxo máximo. Agora,
o ponto mais interessante é como encontrar os ciclos negativos. De fato, a complexidade
do algoritmo é diretamente ligada a esse passo. 

Nesse contexto, chamaremos de forma genérica de \emph{Cycle Canceling} o algoritmo
que a cada passo encontra um ciclo arbitrário. E chamaremos de \emph{Mean Cycle Canceling} 
o algoritmo que a cada passo encontra o ciclo de média mínima.


\section{Implementação}
Todos os algoritmos foram implementados em \texttt{C++} por ser uma linguagem
de alto desempenho e por ser um denominador comum de conhecimento entre os 
integrantes do grupo.

Nas seções abaixo detalhamos a implementação dos algoritmos citados.

\subsection{Cycle Canceling}
A implementação do \emph{Cycle Canceling} assume os seguintes fatos sobre a rede 
de entrada:

\subsubsection{Restrições}

\begin{enumerate}
\item \emph{Todas capacidades, custos e demandas são números inteiros}. Como esses valores
são guardados em variáveis do tipo inteiro de \texttt{C++} os seus valores 
são devem estar no intervalo \emph{aberto} $[-2^{31}, 2^{31})$. Apesar desse intervalo,
as capacidades e custos na nossa implementação devem ser inteiros positivos.

\item \emph{Não existem arcos paralelos}. A capacidade, fluxo e custos são mantidos 
em uma matriz. Dessa forma, como precisamos guardar os arcos reversos não podem 
existir arcos paralelos.  

\item \emph{A rede é direcionada}. Essa é uma restrição do modelo. De forma geral, 
podemos simplesmente transformar uma rede não-direcionada em uma direcionada. Porém,
essa transformação cria arcos paralelos que não são permitidos na nossa implementação.

\end{enumerate}

As restrições mais importantes citatas mais algumas estruturais fazem 
parte dos possíveis códigos de erro do programa: 

\begin{verbatim}
     MCF_SUCCESS - Fluxo a custo mínimo calculado com sucesso. 
     MCF_ERROR_PARALLEL_EDGE - Arestas paralelas na rede entrada.
     MCF_ERROR_BAD_INBALANCE - A soma das demandas mais ofertas não é zero.
     MCF_ERROR_NEGATIVE_EDGE_COST - Alguma aresta tem custo negativo.
     MCF_ERROR_NO_FEASEABLE_SOLUTION - A rede não apresenta uma solução viável.
\end{verbatim}

\subsubsection{Solução viável}

Para achar a solução viável inicial resolvemos um problema de fluxo máximo. 
Criamos dois novos nós \texttt{SINK} e \texttt{SOURCE} e para todo 
vértice de $i$, se $i$ for um nó de oferta, criamos uma aresta do nó \texttt{SOURCE}
para ele com capacidade igual a $b_i$. Por outro lado, se $i$ for um nó de demanda 
criamos uma aresta dele para o \texttt{SINK} com capacidade $-b_i$. 

Uma vez feita a transformação acima, rodamos o algoritmo de \emph{Ford-Fulkerson} para
achar o fluxo máximo entre o \texttt{SOURCE} e o \texttt{SINK}. Se o fluxo máximo tiver
valor $0$, retornamos o código de erro \verb|MCF_ERROR_NO_FEASEABLE_SOLUTION|, caso contrário
removemos os dois nós criados e continuamos com os próximos passos com a solução viável dada pelo fluxo máximo.

\subsubsection{Ciclo Negativo}

Uma vez com uma solução viável x, devemos saber se existe ao menos um ciclo de custo negativo
na rede. Para fazer isso rodamos o algoritmo de \emph{Bellman-Ford} na rede residual $G_x$ e verificamos se  
há ciclos negativos. O algoritmo indica isso de forma direta com um porém. O Bellman-Ford, originalmente, 
acha os menores caminhos de um de partida para todos os outros. Dessa forma, ele só detecta um ciclo
negativo se esse for alcançável pelo nó de partida. Com isso, a princípio, deveríamos rodar o Bellman-Ford
para cada nó na rede para ter certeza que não há ciclos negativos. Isso adicionaria um custo de $|V|$
a um algoritmo que já não é barato. 

Contudo, existe uma forma mais eficiente para detectar os ciclos
negativos a partir do Bellman. Basta criar um novo nó na rede e ligar esse nó a 
todos os outros nós. É fácil ver que qualquer ciclo negativo na rede é alcançável por esse novo nó.   
Desta forma, podemos simplesmente executar o Bellman a partir desse nó para detectar possíveis 
ciclos.

\subsubsection{Complexidade}
A implementação usando Ford-Fulkerson para achar uma solução viável tem complexidade:

$$
	O (C|V|(|V| + |E|) + |V|^2|E| UC)
$$ 

onde $C$ é a maior capacidade de um arco na rede e $U$ é o maior custo de um arco. 
A parte, $C |V| (|V| + |E|)$ vem da primeira fase de achar uma solução viável.

\subsection{Mean Cycle Canceling}

A implementação do \emph{Mean Cycle Canceling} só difere da anterior na parte de achar o
ciclo de média mínima. 

Uma vez que temos uma solução viável x e uma rede residual $G_x$ queremos
achar o ciclo de média mínima. Para achar o ciclo fazemos uma busca binária 
pelo ciclo média mínima.

De forma detalhada, primeiro observe que podemos
testar se existe um ciclo com de valor até $l$ da seguinte forma: Modifique
todas os custos $c_{ij}$ da rede residual para $c_{ij} - l$. Assim, existe um ciclo de média até $l$
no rede original se, e somente se, existir um ciclo negativo na rede residual modificada.
Podemos novamente utilizar o Bellman-Ford para verificar se há um ciclo negativo. Se existir 
um ciclo, tente valores menores para $l$, caso contrário, tentar valores maiores.  
 
Como o valor de $l$ é limitado por $-U \le l \le U$, achar o ciclo mínimo custará 
$O (log(U) * |V|^2)$. Como é provado que o número de interações máximo desse algoritmo
é um polinômio, a complexidade total será polinomial. 

\section{Instâncias}

\section{Experimentos}

\section{Conclusão}


\bibliographystyle{plain}

\bibliography{paper}


\end{document}
